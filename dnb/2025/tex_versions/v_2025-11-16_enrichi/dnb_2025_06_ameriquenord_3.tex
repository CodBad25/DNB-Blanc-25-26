
On considère les deux programmes de calcul suivants :

\begin{center}
\begin{tabularx}{\linewidth}{m{6.cm}|X}
\textbf{Programme A}&\textbf{Programme B}\\
\begin{itemize}[label= $\bullet~~$]
\item Choisir un nombre
\item Multiplier par 3
\item Ajouter 15
\item Diviser par 3
\item Soustraire le nombre de départ
\end{itemize}&
\vspace*{-2cm}
\includegraphics[width=0.8\linewidth]{images/dnb_2025_06_ameriquenord_3_Image_007.gif}
\end{tabularx}
\end{center}

\begin{enumerate}
\item Montrer que, lorsque le nombre choisi est 4, le résultat obtenu avec le programme A est 5.
\item Montrer que, lorsque le nombre choisi est $- 2$, le résultat obtenu avec le programme A est 5.
\item Justifier que l'affirmation suivante est vraie :

\begin{center}\og Le programme A donne toujours le même résultat. \fg\end{center}

\item Lorsque le nombre choisi est 10, quel résultat obtient-on avec le programme B ?
\item Il existe exactement deux nombres pour lesquels les programmes A et B fournissent à chaque fois des résultats identiques.

Quels sont ces deux nombres?
\end{enumerate}


